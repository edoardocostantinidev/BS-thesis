Questo progetto di tesi ha visto la nascita di un nuovo tipo di solutore specializzato nella simulazione di processi LMD che utilizza una combinazione
di metodi, Lagrangiano e Euleriano, per predire il movimento di un flusso di particelle metalliche. Nonostante i risultati in grado di computare adesso
sono numericamente corretti vi sono ancora alcune funzionalità la cui implementazione aumenterebbe certamente il valore del simulatore.
Una fra queste sicuramente è l'introduzione di ulteriori algoritmi di parallelizzazioni per garantire tempi di simulazione ancora più bassi. 
\section{Ulteriori parallelizzazioni}\label{DistribuzioneTriangulation}
Al momento viene utilizzato un approccio ibrido per quanto riguarda la distribuzione del carico di lavoro della simulazione su più processi.
In particolare nella fase di computazione Euleriana dei flussi viene impiegato un algoritmo di parallelizzazione di calcolo (\ref*{parallelizzazione}).
Dealii, però, offre diversi strumenti per parallelizzare efficentemente ulteriori carichi di lavoro, tra questi c'è la possibilità di distribuire la 
mesh, tramite l'oggetto \texttt{parallel::distributed::Triangulation< dim, spacedim >} (\ref*{mesh}) su più calcolatori per permettere di parallelizzare 
le operazioni di costruzione o suddivisione nel caso di mesh molto risolute \cite{BBHK11}.

Un altro spunto che vale la pena considerare è quello di parallelizzare il metodo Lagrangiano responsabile per la computazione del flusso di particelle metalliche.
Questo tipo di processo accoppiato ad alcuni algoritmi accennati nella sezione \ref*{tbb}, offerti dalla libreria Intel TBB, potrebbero ridurre drasticamente i tempi di calcolo.
Quest'implementazione certamente richiederebbe uno sforzo implementativo considerevole in quanto ogni struttura dati andrebbe correttamente gestita per evitare race-condition e thread-lock.

Infine il simulatore gioverebbe sicuramente da un interfaccia grafica capace di gestire correttamente i vari file di input e parametri, cosi da semplificare un futuro utilizzo
che al momento risulta piuttosto ostico a chi non ha partecipato allo sviluppo. Si potrebbe realizzare un semplice wrapper che organizza i file e gestisce l'esecuzione del simulatore
in maniera autonoma.