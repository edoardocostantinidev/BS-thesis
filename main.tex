%
% Esempio di uso della class phddiauniroma3
% (C) 2006-2008, Franco Milicchio et al.
%

\documentclass{TesiDiaUniroma3}

\usepackage{listings}
\usepackage{xcolor}
\usepackage{amssymb}
\usepackage{float}
\definecolor{light-gray}{gray}{0.95}
\usepackage{mdframed} %nice frames

% --- INIZIO dati relativi al template TesiDiaUniroma3
% dati obbligatori, necessari al frontespizio
\titolo{Simulazione Fluidodinamica di processi di produzione additiva}
\autore{Edoardo Costantini}
\matricola{500130}
\relatore{Ing. Franco Milicchio}
\annoAccademico{2019/2020}
\dedica{A chi non ha mai smesso di credere in me.}

% --- INIZIO richiamo di pacchetti di utilità. Questi sono un esempio e non sono strettamente necessari al modello per la tesi.
\usepackage[plainpages=false]{hyperref}	% generazione di collegamenti ipertestuali su indice e riferimenti
\usepackage[all]{hypcap} % per far si che i link ipertestuali alle immagini puntino all'inizio delle stesse e non alla didascalia sottostante
\usepackage{amsthm}	% per definizioni e teoremi
\usepackage{amsmath}	% per ``cases'' environment
\usepackage{textcomp}
% --- FINE riachiamo di pacchetti di utilità

\begin{document}
% ----- Pagine di fronespizio, numerate in romano (i,ii,iii,iv...) (obbligatorio)
\frontmatter
\generaFrontespizio
\generaDedica
\ringraziamenti{ringraziamenti}	% inserisce i ringraziamenti e li prende in questo caso da ringraziamenti.tex
\introduzione{introduzione}		% inserisce l'introduzione e la prende in questo caso da introduzione.tex
\generaIndice
\generaIndiceFigure

% ----- Pagine di tesi, numerate in arabo (1,2,3,4,...) (obbligatorio)
\mainmatter
\capitolo{Manifattura additiva}{capitoli/capitolo1/capitolo1}
\capitolo{Modello Matematico}{capitoli/capitolo2/capitolo2}
\capitolo{Computazione Fluidodinamica}{capitoli/capitolo3/capitolo3}
\capitolo{Background Tecnologico}{capitoli/capitolo4/capitolo4}
\capitolo{Solutore Computazionale}{capitoli/capitolo5/capitolo5}
\capitolo{Risultati}{capitoli/capitolo6/capitolo6}

% ----- Parte finale della tesi (obbligatorio)
\backmatter
\conclusioni{conclusioni}

% Bibliografia con BibTeX (obbligatoria)
% Non si deve specificare lo stile della bibliografia
\bibliography{bibliografia} % inserisce la bibliografia e la prende in questo caso da bibliografia.bib

\end{document}
