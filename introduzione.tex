Per stare al passo con la sempre più esigente richiesta dei sistemi manufatturieri moderni, vi è la necessità di introdurre nuovi processi di
manufattura capaci di rispondere prontamente a tale richiesta. In questo contesto la manifattura additiva offre diversi vantaggi rispetto i più tradizionali
processi. Tra questi notiamo soprattutto la capacità di realizzare complessi componenti in termini di geometrie e materiali utilizzati. Proprio questa
libertà di operazione rende la manifattura additiva responsabile anche di un \textit{time-to-market} notevolmente ridotto. La manifattura additiva rende
il processo di creazione dei componenti totalmente digitale, scavalcando numerosi problemi della versione più tradizionale, come per esempio
logistica e stoccaggio delle parti necessarie. Di conseguenza i costi di una produzione basata su manifattura additiva sono relativamente ridotti
La manifattura additiva introduce anche vantaggi in ambito riparazioni di precisioni, abbassando i costi operando direttamente su componenti danneggiati 
senza il bisogno di sostituzioni. Si può trarre beneficio di questi vantaggi sia se si opera in ambiti di piccola produzione che di prototipazione.
La manifattura additiva è un ambito di ricerca in rapida evoluzione poiché diversi settori traggono benefici da queste tecniche additive, tra questi ricordiamo
il settore automobilistico, aereospaziale, e biomedico. 
\section{Obiettivo}\label{obiettivo}
L'obiettivo di questo progetto di tesi è quello di riuscire a simulare correttamente, dal punto di vista numerico, uno scenario di manifattura additiva
attraverso metodo di Laser Metal Deposition. Lo scenario prevede l'emissione da parte di un ugello di particelle metalliche schermate da un gas inerte
che previene l'ossidazione su una superficie obiettivo; con particolare interesse verso l'interazione delle particelle con il flusso di gas, in termini di temperatura e pressione, e con le pareti
dell'ugello, in termini di deflezioni. Perseguendo questo obiettivo è stato sviluppato un simulatore numerico capace di sfruttare metodi Lagrangiani per il calcolo delle traiettorie delle particelle metalliche
e metodi Euleriani per il calcolo delle proprietà del flusso di gas. Il simulatore sviluppato è inoltre in grado di produrre dei file interpretabili da popolari strumenti di visualizzazione
come mostrato nel capitolo \ref*{cap:capitoli/capitolo6/capitolo6}.

\section{Capitoli}\label{capitoli}
Di seguito una breve introduzione di ogni capitolo della presente tesi.
\paragraph{Capitolo 1.} Introduzione ad alto livello dei vari processi di manifattura additiva odierni e una vista più dettagliata
del processo LMD oggetto di questa tesi. Questo capitolo può considerarsi una estensione dell'introduzione. 
\paragraph{Capitolo 2.} Presentazione di alcuni concetti matematici fondamentali per la comprensione della tesi. In particolare vengono
introdotti argomenti quali: equazioni differenziali alle derivate parziali, equazioni di Navier-Stokes, metodo degli elementi finiti, metodo Newthon-Raphson. 
Il capitolo va inteso come una contestualizzazione e introduzione dei concetti, non vi sono quindi dimostrazioni. 
\paragraph{Capitolo 3.} Visione del settore della CFD e dei concetti chiave che lo caratterizzano. In questo capitolo vi è anche
un dettaglio sui metodi Lagrangiani-Euleriani utilizzati per sviluppare il solutore oggetto della tesi. Infine alcuni esempi di applicazioni comuni per 
dove le simulazioni CFD trovano largo impiego.  
\paragraph{Capitolo 4.} Il solutore sviluppato fa uso di svariati strumenti tecnologici che hanno influenzato, sia positivamente che non, il lavoro svolto; questo capitolo
descrive questi strumenti e il loro utilizzo all'interno della tesi.
\paragraph{Capitolo 5.} Il quinto capitolo contiene una descrizione dettagliata del funzionamento e dell'architettura del solutore sviluppato, oggetto della tesi.
Vengono presentati e descritti tutti i moduli che compongono il solutore, essi sono organizzati uno per sezione. 
\paragraph{Capitolo 6.} Questo capitolo espone i risultati raggiunti con il solutore implementato, fornendo diverse figure e render descrittive di quanto è stato fatto.
I risultati sono raggruppati per area di interesse e infine mostrati nel loro insieme.
\paragraph{Capitolo 7.} Il capitolo conclusivo descrive brevemente ciò che è stato raggiunto. Vengono inoltre discussi alcuni
futuri sviluppi riguardo il progetto svolto.
